\documentclass[11pt]{book}
\usepackage[utf8]{inputenc}
\usepackage[dvipsnames]{xcolor}
\usepackage{amsmath,amsthm,amssymb}
\usepackage{mathtools}
\usepackage{tcolorbox}
\usepackage{fancyhdr}
\usepackage{hyperref}
\usepackage[dvipsnames]{xcolor}
\usepackage{graphicx}
\usepackage{dirtytalk}
\usepackage{csquotes}
\usepackage{epsfig}
\usepackage[b5paper, inner=2cm,outer=2cm,top=3cm,bottom=3cm]{geometry}
\usepackage{asymptote}
\usepackage[Glenn]{fncychap}

\pagestyle{fancy}
\fancyhf{}
\setlength{\headheight}{13.59999pt}
\addtolength{\topmargin}{-1.59999pt}
\fancyhead[R]{\jobname}
\fancyhead[L]{ Pun Tresattayapan\textcolor{blue!50!white}{\href{https://www.facebook.com/puntre.puntre/}{\nolinkurl{*}}}(\today)}
\fancyfoot[R]{\thepage}


\renewcommand{\headrulewidth}{0.5pt}


\title{\jobname}
\author{Pun Tresattayapan}
\date{\today}





\DeclareMathOperator{\ord}{ord}
\DeclareMathOperator{\lcm}{lcm}


\begin{document}

\definecolor{ltblue}{HTML}{14DDEC}
\definecolor{ltred}{HTML}{FFB6C1}
\definecolor{ltgreen}{HTML}{50C878}

\newcommand{\cbrt}[1]{\sqrt[3]{#1}}
\newcommand{\floor}[1]{\left\lfloor #1 \right\rfloor}
\newcommand{\ceiling}[1]{\left\lceil #1 \right\rceil}
\newcommand{\paren}[1]{\left(#1\right)}
\newcommand{\bparen}[1]{\left[ #1 \right]}
\newcommand{\abs}[1]{\left|#1\right|}
\newcommand{\ol}{\overline}
\newcommand{\ul}{\underline}
\newcommand{\wt}{\widetilde}
\newcommand{\wh}{\widehat}
\newcommand{\eps}{\varepsilon}

\newcommand{\mangle}{\measuredangle}


\newcommand{\btx}[1]{\textcolor{blue!70}{#1}}
\newcommand{\rtx}[1]{\textcolor{red!70}{#1}}
\newcommand{\bbf}[1]{\textcolor{blue!70}{\textbf{#1}}}
\newcommand{\rbf}[1]{\textcolor{red!70}{\textbf{#1}}}

\makeatletter
\renewenvironment{proof}[1][\proofname]{\par
	\pushQED{\qed}%
	\normalfont \topsep6\p@\@plus6\p@\relax
	\trivlist
	\item\relax
	{\itshape
		#1\@addpunct{.}}\hspace\labelsep\ignorespaces
    \setcounter{equation}{0}
}{%
	\popQED\endtrivlist\@endpefalse\vspace{-13 pt}\noindent\strut\hrulefill \strut \smallskip \newline
}
\makeatother
\makeatletter
\newenvironment{subproof}[1][\proofname]{\par
	\pushQED{\qed}%
	\normalfont \topsep6\p@\@plus6\p@\relax
	\trivlist
	\item\relax
	{\itshape
		#1\@addpunct{.}}\hspace\labelsep\ignorespaces
}{%
	\popQED\endtrivlist\@endpefalse
}
\makeatother


%% cyc and sym sum, prod

\newcommand{\cycsum}{\sum_{\mathrm{cyc}}}
\newcommand{\symsum}{\sum_{\mathrm{sym}}}
\newcommand{\cycprod}{\prod_{\mathrm{cyc}}}
\newcommand{\symprod}{\prod_{\mathrm{sym}}}

%% Sets

\newcommand{\bbN}{\mathbb{N}}
\newcommand{\bbZ}{\mathbb{Z}}
\newcommand{\bbZp}{\mathbb{Z^+}}
\newcommand{\bbZm}{\mathbb{Z^-}}
\newcommand{\bbC}{\mathbb{C}}
\newcommand{\bbR}{\mathbb{R}}

%% head & section

\renewcommand{\headrulewidth}{0.5pt}

\makeatletter
%% See pp. 26f. of 'The LaTeX Companion,' 2nd. ed.
\def\@seccntformat#1{\@ifundefined{#1@cntformat}%
    {\csname the#1\endcsname\quad}%      default
    {\csname #1@cntformat\endcsname}}%   individual control
    \newcommand{\section@cntformat}{\textcolor{blue!70}{$\spadesuit$\hspace{0.1cm}\thesection}\quad}
    \newcommand{\subsection@cntformat}{\textcolor{red!70}{$\clubsuit$\hspace{0.1cm}\thesubsection}\quad}
    \newcommand{\subsubsection@cntformat}{\textcolor{green!75!blue}{$\diamondsuit$\hspace{0.1cm}\thesubsubsection}\quad}
    \newcommand{\subsubsubsection@cntformat}{\textcolor{green!75!blue}{$\heartsuit$\hspace{0.1cm}\thesubsubsection}\quad}
%\newcommand{\paragraph@cntformat}{\S\theparagraph\quad}
%\newcommand{\subparagraph@cntformat}{\S\thesubparagraph\quad}
\makeatletter

%% theoremstyles (again, some are edited from Dylandi)

\newtheoremstyle{puntheo}%                % Name
  {0pt}%                                     % Space above
  {2pt}%                                     % Space below
  {}%                                     % Body font
  {}%                                     % Indent amount
  {\color{red!70}\bfseries}%                            % Theorem head font
  {.}%                                    % Punctuation after theorem head
  { }%                                    % Space after theorem head, ' ', or \newline
  {\thmname{#1}\thmnumber{ #2}\thmnote{ (#3)}}% 

\newtheoremstyle{punthm}%                % Name
  {0pt}%                                     % Space above
  {2pt}%                                     % Space below
  {}%                                     % Body font
  {}%                                     % Indent amount
  {\color{blue!70}\bfseries}%                            % Theorem head font
  {.}%                                    % Punctuation after theorem head
  { }%                                    % Space after theorem head, ' ', or \newline
  {\thmname{#1}\thmnumber{ #2}\thmnote{ (#3)}}%                                     % Theorem head spec (can be left empty, meaning `normal')

  
\newtheoremstyle{punrem}%                % Name
  {0pt}%                                     % Space above
  {2pt}%                                     % Space below
  {}%                                     % Body font
  {}%                                     % Indent amount
  {\color{black}\itshape}%                            % Theorem head font
  {.}%                                    % Punctuation after theorem head
  { }%                                    % Space after theorem head, ' ', or \newline
  {\thmname{#1}\thmnumber{ #2}\thmnote{ (#3)}}%                                     % Theorem head spec (can be left empty, meaning `normal')


  \newtheoremstyle{punnote}%                % Name
  {0pt}%                                     % Space above
  {2pt}%                                     % Space below
  {}%                                     % Body font
  {}%                                     % Indent amount
  {\color{blue!70}\itshape}%                            % Theorem head font
  {.}%                                    % Punctuation after theorem head
  { }%                                    % Space after theorem head, ' ', or \newline
  {\thmname{#1}\thmnumber{ #2}\thmnote{ (#3)}}%                                     % Theorem head spec (can be left empty, meaning `normal')

\newtheoremstyle{punexercise}%                % Name
  {0pt}%                                     % Space above
  {2pt}%                                     % Space below
  {}%                                     % Body font
  {}%                                     % Indent amount
  {\color{blue!70}\itshape}%                            % Theorem head font
  {.}%                                    % Punctuation after theorem head
  { }%                                    % Space after theorem head, ' ', or \newline
  {\thmname{#1}\thmnumber{ #2}\thmnote{ (#3)}}%                                     % Theorem head spec (can be left empty, meaning `normal')



%% uses of theorem styles

\theoremstyle{puntheo}
\newtheorem{thm}{Theorem}[section]

\theoremstyle{punthm}
\newtheorem{lem}[thm]{Lemma}


\theoremstyle{punthm}
\newtheorem{prop}[thm]{Proposition}

\theoremstyle{punthm}
\newtheorem*{cor}{Corollary}

\theoremstyle{punthm}
\newtheorem{defn}{Definition}[section]

\theoremstyle{punthm}
\newtheorem{exmp}{Example}[section]

\theoremstyle{punthm}
\newtheorem{pro}{Problem}[section]

\theoremstyle{punthm}
\newtheorem{proe}{Problem}

\theoremstyle{punrem} 
\newtheorem*{rem}{Remark}

\theoremstyle{punexercise} 
\newtheorem{exer}{Exercise}[section]

%% choices

\newcounter{choice}
\renewcommand\thechoice{\Alph{choice}}
\newcommand\choicelabel{\thechoice.}

\newenvironment{choices}%
  {\list{\choicelabel}%
     {\usecounter{choice}\def\makelabel##1{\hss\llap{##1}}%
       \settowidth{\leftmargin}{W.\hskip\labelsep\hskip 2.5em}%
       \def\choice{%
         \item
       } % choice
       \labelwidth\leftmargin\advance\labelwidth-\labelsep
       \topsep=0pt
       \partopsep=0pt
     }%
  }%
  {\endlist}

\newenvironment{oneparchoices}%
  {%
    \setcounter{choice}{0}%
    \def\choice{%
      \refstepcounter{choice}%
      \ifnum\value{choice}>1\relax
        \penalty -50\hskip 1em plus 1em\relax
      \fi
      \choicelabel
      \nobreak\enskip
    }% choice
    % If we're continuing the paragraph containing the question,
    % then leave a bit of space before the first choice:
    \ifvmode\else\enskip\fi
    \ignorespaces
  }%
  {}


  %% box UNFINISHED (Add other colors)
\newcommand{\lbBox}[2]{\begin{tcolorbox}
		[toprule=-1mm, colback=ltblue!10!white, colframe=ltblue!95!black, sharp corners, title=\textbf{#1}]
	#2
	\end{tcolorbox}}

\newcommand{\lrBox}[2]{\begin{tcolorbox}
		[toprule=-1mm, colback=ltred!10!white, colframe=ltred!95!black, sharp corners, title=\textbf{#1}]
	#2
	\end{tcolorbox}}



\newcommand{\Thm}[2]{\lrBox{\thm{\textbf{#1}}}{#2}}

\newcommand{\Lem}[2]{\lrBox{\lem{\textbf{#1}}}{#2}}

\newcommand{\Defn}[2]{\lbBox{\defn{\textbf{#1}}}{#2}}

\newcommand{\Exmp}[2]{\lbBox{\exmp{\textbf{#1}}}{#2}}

\newcommand{\Pro}[2]{\lbBox{\pro{\textbf{#1}}}{#2}}

\newcommand{\Proe}[2]{\lbBox{\proe{\textbf{#1}}}{#2}}

\newcommand{\Exer}[2]{\lbBox{\exer{\textbf{#1}}}{#2}}

	

\newcommand{\lbbox}[1]{\begin{tcolorbox}
	[colback=ltblue!10!white, colframe=ltblue!80!black, sharp corners]
	#1
	\end{tcolorbox}}



\newcommand{\claim}[1]{\begin{tcolorbox}
	[colback=ltgreen!10!white,colframe=ltgreen!90!green,sharp corners,arc=0mm,boxrule=0mm,leftrule=0.7mm]
	\textcolor{ltgreen}{\textbf{Claim.}} #1
		\end{tcolorbox}}%		claim

\newcommand{\lemma}[1]{\begin{tcolorbox}
	[colback=ltgreen!10!white,colframe=ltgreen!90!black,sharp corners,arc=0mm,boxrule=0mm,leftrule=0.7mm]
	\textcolor{ltgreen}{\textbf{Lemma.}} #1
		\end{tcolorbox}}%		claim

\newcommand{\Remark}[1]{\begin{tcolorbox}
	[colback=black!10!white,colframe=black!60!black,sharp corners,arc=0mm,boxrule=0mm,leftrule=0.7mm]
	\rem{#1}
		\end{tcolorbox}}%		Remark



%===========================================================================
%======================= typesetting start here ============================
%===========================================================================
\maketitle
{\color{ltblue!70!blue}\tableofcontents}

\include{PHP.tex}
\end{document}
